% !TeX root = ../main.tex

\ustcsetup{
  keywords = {
    随机生成森林模型,临界现象,蒙特卡洛方法,动态连接数据结构 
  },
  keywords* = {
    Spanning-forest model, Critical phenomena, Monte Carlo method, Dynamic connectivity structure
  },  
}

\begin{abstract}
  随机生成森林作为一个统计模型具有丰富的临界现象。在高维情况下,体系将经历一逾渗相变,但区别于标准的的逾渗模型,
  该模型的超临界相具有更加复杂的结构,最近的理论预测该模型的超临界相中具有类似临界点处的标度行为。针对该问题,本文通过蒙特卡洛模拟对完全图
  和三维正方晶格下的随机生成森林模型的临界点和超临界相进行了探究。

  第一章是关于研究模型和研究方法的总述。我们首先介绍了模型的历史背景,并总结了现有的理论和数值工作。之后文章介绍了蒙特卡洛方法在统计
  物理中的应用,叙述了这类方法的基本思想和数学原理。

  第二章详细介绍了随机生成森林模型的相关理论和已有结论。
  % 利用Fortuin-Kasteleyn表象,我们展示了该模型可作为$q$态Potts模型的$q \to 0$ 极限情况。之后
  我们给出了生成森林模型的费米场表示,并将原模型的两点连接概率与费米表示中的两点关联函数相联系,从而给出了模型在维度$d \geq3$
  的超临界相的行为。
  % 最后,文章概括了模型在完全图情况的结论,包括集团在临界点和超临界相处的标度行为。

  第三章描述了模拟中用到的算法和数据结构。首先介绍了用于更新生成森林构型的Sweeny算法,针对算法中用到的动态连接性询问操作,我们进一步提出利用动态树
  数据结构来降低算法的时间复杂度。

  第四章为对模拟结果的分析。通过分析集团尺寸以及分布的标度行为,我们证实了体系的超临界相中仍然存在临界行为,并且对于三维情况,超临界相中的
  临界指数和临界点处的指数不同。另外,数据表明,有限大系统模型的构型空间可以分为两个具有不同标度行为的部分,其中一个部分对应于模型在热力学
  极限下的行为,而另一个部分作为有限尺度的效应随着体系尺寸增大其体积趋于零。

\end{abstract}

\begin{abstract*}
The random spanning forests is a prototypical model in statistical physics due to its rich critical phenomena.
It has been shown that the model goes through a percolation transition in dimension $d\geq 3$. In contrast to the standard bond percolation, the model
is conjectured to exhibit critical scaling behaviors even in its supercritical phase. To verify the conjecture, we carry on a Monte
Carlo simulation of the model on the complete and cubic lattice at its critical point and supercritical phase.

In Chap.~\ref{chap:intro}, we give a brief review of the history and summarize the previous works.
Then we introduce the Monte Carlo methods, an important numerical technique in physics, by elucidating its 
central ideas and general framework.

In Chap.~\ref{chap:SPF}, the fermionic representation of the spanning-forest model is derived, from which one can establish a relation between the
two-point connectivity in the original model and the two-point correlation in the fermionic model. We then summarize the results
obtained through this connection.

In Chap.~\ref{chap:Algo}, we introduce the Sweeny algorithm, which is used to simulate the spanning forests in this work. Besides, we propose
to leverage the link-cut trees data structure to speed up the dynamic connectivity query in the algorithm.

In Chap.~\ref{chap:results}, the simulation results are presented. Through finite-size analysis, we confirm that there exist
critical scaling behaviors entailed in the supercritical phase. Besides, for the cubic lattice, we find that the critical exponents
that characterize the scaling in the supercritical phase are different from the exponents at criticality. Furthermore,
for finite systems, our data detect two different sectors in the configuration space of the model, which exhibit distinct
scaling behaviors. One of the sectors is responsible for the model's behaviors in the infinite-volume limit, while the other
is a finite-size effect, which decays to zero as the system size grows.

\end{abstract*}
