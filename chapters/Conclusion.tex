\chapter{Conclusions}\label{chap:conclusion}

We studied the spanning-forest model using the Sweeny algorithm. To speed up the dynamical connectivity query 
used in the algorithm, we implement an advanced data structure, the link-cut trees, which reduces the amortized time complexity
for update and query to $O(\log V)$, where $V$ is the number of vertices in the underlying lattice. 

We simulated the model on the complete graph and the cubic lattice with periodic boundary conditions in its supercritical phase. For the complete graph,
we numerically confirmed that the fractal dimension of the non-giant clusters in the supercritical phase is $\frac{2}{3}$, identical to the fractal dimension at criticality. Furthermore, we found that, for a finite system with volume $V$, the configuration
space of the spanning forests can be divided into two sectors, denoted as sectors~\Romannum{1} and \Romannum{2}. 
Sector~\Romannum{1} is responsible for the model's scaling behaviors in the infinite-volume limit, in which 
the fluctuations of the largest cluster $\scrC_1$ are $O(V^{2/3})$, and the second-largest cluster $\scrC_2$ scales as $V^{2/3}$. 
Moreover, the cluster-size distribution (excluded $\scrC_1$) $n^\prime(s, V)$ in this sector has the same scaling form as the
total cluster-size distribution $n(s, V)$ at criticality, which further demonstrates the critical properties of the supercritical phase.
In contrast, sector~\Romannum{2} is a finite-size effect, which vanishes asymptotically as $V^{-1/2}$. In this sector, 
cluster $\scrC_1$ has $O(V)$ fluctuations, and $\scrC_2$ is proportional to the system volume. For the cubic lattice, 
we found that the non-giant clusters in the supercritical phase are still fractals, whose fractal dimension is estimated to be $2.30(3)$, 
which is different from the fractal dimension $d_\text{f} = 2.5838(6)$ at the model's critical point. Similar to the complete graph, 
we also found two distinct sectors in the configuration space for 3D. Sector \Romannum{1} dominates in the $L\to\infty$ limit,
and both the fluctuations of $\scrC_1$ and the scale of $\scrC_2$ are described by fractal dimension $d_\text{f} = 2.30(3)$. Besides,
the cluster-size distribution (excluded $\scrC_1$) $n^\prime(s, L)$ can be described by the critical scaling hypothesis $n^\prime(s, L) = s^{-\tau}\tilde{n}^\prime(s/L^{d_\text{f}})$
with $\tau = 2.30$ and $d_\text{f} = 2.30$. Sector \Romannum{2}, instead, decays asymptotically as $L^{-0.75}$, and both the fluctuations of $\scrC_1$ and the scale of $\scrC_2$ in this sector
are of the order of system volume.

Finally, let's conclude with some future research directions. In the thesis, we have observed the critical behaviors of the spanning-forest model entailed in its supercritical phase in $d=3$, but
the exponents that are used to characterize the scaling in the supercritical phase are different from the ones at the critical point. 
It's natural to ask whether it is true in higher dimensions, in particular $d=4, 5$ since it has been conjectured in Ref.~\inlinecite{Deng2007} that the upper 
critical dimension of the model is $d=6$. Another potential direction is to examine whether the supercritical phase of the spanning forests
belongs to the universality class of the spanning-tree model, which is the $w\to \infty$ limit of the spanning forests. 
Clarifying this question is important in understanding the renormalization flows in the model's phase diagram. Last but not the least,
it would also be interesting to explore the geometric structures of the clusters in the supercritical phase.
