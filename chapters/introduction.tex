\chapter{Introduction}\label{chap:intro}

\section{Random spanning forests}
The random spanning forests is a particularly interesting model in statistical physics due to its special critical phenomena
and close connections to other statistical models. It decsribes the set of spanning forests $\scrF$, i.e., acyclic spanning subgraphs, of the underlying
lattice $G$ with each occupied edge of the forest weighted by $w$. The partition function of the model is written as
\begin{equation}\label{eq:SPF_partition}
	\scrZ = \sum_{F \in \scrF} w^{|F|},
\end{equation}
where $|F|$ counts the number of occupied edges in $F$.

The model is reminiscent of the famous model of bond percolation, where each bond is occupied independently with probability $p$ and do not need to form a forest.
In percolation, it's well known that for dimension $d \geq 2$, there exists a critical point $0 < p_c < 1$ such that there is a unique giant cluster whose size is of the order of the system volume
at the supercritical phase $p > p_c$. In contrast, Roland \textit{et al.} \cite{Roland2021Percolation} have shown that the percolation of spanning forests happens only for $d \geq 3$ 
because of the geometrical constraint of the model. They also conjectured that the supercritical phase of the spanning-forest model is more interesting in terms of the scaling behaviors of non-giant clusters (trees). 
The random spanning forests can also be viewed as special case of the $q$-state Potts model \cite{Potts1952, Wu1982}, another prototypical model in the study of phase transitions and critical phenomena.
At first, the $q$-state Potts model was proposed to describe ferromagnets on a crystalline lattice, and $q$ therefore takes integer values.
In 1972, a seminal work by Fortuin and Kasteleyn \cite{Fortuin1972} showed that the Potts model could be mapped to the random cluster model, in which $q$ can be 
generalized to any arbitrary complex number. In particular, as illustrated in Chap.~\ref{chap:SPF}, by taking the limit of $q\to 0$ appropriately,
the random cluster model recovers to the spanning forests.

The theoretical studies of the spanning-forest model started with the paper by Caracciolo \textit{et al.} \cite{Caracciolo2004}, in which they derived
an equivalent fermionic theory and showed that the theory could further be mapped onto a non-linear sigma model taking vales in the unit supersphere
$\mathbb{R}^{1|2}$ ($\text{OSp}(1|2)$-invariant sigma model) to all orders in perturbation theory. 
Later, Deng \textit{et al.}\cite{Deng2007} presented Monte Carlo simulations of the model in spatial dimensions $d = 3, 4, 5$ and demonstrated that, for each case, there is a second-order phase transition at a finite weight
$w_c$. They further conjectured that the upper critical dimension is $d=6$. Recently, the authors in Ref.~\inlinecite{Roland2021Hyperbolic} extended the results
of Caracciolo \textit{et al.} and reinterpreted the fermionic theory as a non-linear sigma model with hyperbolic target space $\mathbb{H}^{0|2}$, based on which
they proved that there is no phase transition in two dimensions using a Mermin–Wagner argument. With this formulation, they further
obtained \cite{Roland2021Percolation} that a percolation transition is present for $d \geq 3$, consistent with the previous numerics. Despite these developments, there remain several aspects of
the model worth exploring. In the thesis, we focus on the finite-size scaling behaviors of the non-giant clusters at the supercritical phase.
It has been proved that \cite{Luczak1992,Martin2018} when $G$ is a complete graph ($d=\infty$) and the system is at the supercritical phase, there are an unbounded number of clusters with sizes of the order of $|V|^{2/3}$ apart from the macroscopic component,
where $|V|$ is the system size. Remarkably, the scale of clusters at criticality, in this case, is also $|V|^{2/3}$. In other words, the supercritical
phase without the largest component looks like criticality. Our goal is to numerically examine whether this phenomenon happens at finite dimensions, in particular, $d=3$.

\section{Monte Carlo methods}
We use Monte Carlo methods to study the spanning-forest model. These methods play a very important role in statistical physics by
providing a way to extract statistical properties from high dimensional configuration space. To be more specific, in statistical
physics, we are usually interested in the ensemble average of various observables of system of equilibrium, which is governed by a probability distribution $\pi$ over the configuration space $\scrS$ of the system. 
For example,  the ensemble average of observable $f(x)$ (as a function of state $x$) is defined by
\begin{equation}
	\langle f \rangle = \sum_{x \in \scrS} f(x) \pi(x).
\end{equation}
Typically, the state space $\scrS$ is extremely large such that directly computing the weighted average by enumerating all possible states is impractical.
The central idea of Monte Carlo simulations is to circumvent this difficulty by importance sampling, i.e., designing a Markov process to directly sample state from the distribution
$\pi$, and then compute the unweighted average as the estimated value of the ensemble average, i.e.,
\begin{equation}
	\bar{f} = \frac{1}{N}\sum_{i=1}^N f(x_i) \approx \langle f \rangle,
\end{equation}
where the number of sampling steps $N$ can be much smaller than the dimension of the state space $\scrS$. 
Moreover, one can also determine the uncertainties of the estimate by various techniques like binning method.
We will elaborate on the algorithm and data structure used to simulate the spanning forests in Chap.~\ref{chap:Algo}.

\section{Plan of the thesis}
The thesis is organized as follows. In Chap.~\ref{chap:SPF}, we eclucidate the relation between the random spanning forests and
Potts model and derive the corresponding fermionic representation. Then we summarize the analytical results in Ref.~\inlinecite{Roland2021Hyperbolic,Roland2021Percolation}.
The Monte Carlo algorithm and data structure implemented in this work is covered in Chap.~\ref{chap:Algo}. In Chap.~\ref{chap:results}, we analyze the simulation results, and a conclusion is given in Chap.~\ref{chap:conclusion}.


% --------- for chapter 2 -----------------
% which has the following partition function
% \begin{equation}\label{eq:RC_partition}
% 	\scrZ_{\textsc{SC}}	 = \sum_{\omega} q^{C(\omega)} p^{|\omega|}(1-p)^{|G| - |\omega|},
% \end{equation}
% where $\sum_\omega$ sums over all spanning subgraphs of the underlying lattice $G$, $C(\omega)$ denotes the number of connected
% components in $\omega$, and $p$ is a parameter related to the nearest-neighbor coupling in the original $q$-state Potts model.
% For $q $


% defined on a graph $G$ is a statistical model that desrcribes the set of 
% (unrooted) acylic spanning subgraphs of $G$, with each occupied edge has weight $w$.
