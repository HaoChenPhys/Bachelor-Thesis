\chapter{The spanning-forest model}\label{chap:SPF}

\section{Potts model and spanning forests}
In this section, we will explain how the spanning-forest model arises in the $q\to 0$ limit of the $q$-state Potts model.
Given a finite lattice $G = (V, E)$, the $q$-state Potts model is defined as follows: for each site $i\in V$, the spin 
$\sigma_i$ takes values from $\{0, \dots, q-1\}$, and the Hamiltonian for spin configuration $\sigma \in \Sigma = \{0,\dots, q-1\}^{|V|}$
is written as
\begin{equation}\label{eq:Potts_H}
	\scrH(\sigma) = -\sum_{\langle ij \rangle \in E}J \delta_{\sigma_i, \sigma_j},
\end{equation}
where $J>0$ is the ferromagnetic coupling constant. For convenience, we set $J = q$ in the following discussion.
% Let $\Sigma = \{0,\dots, q-1\}^{|V|}$ denote the spin configuration space on $G$.
The corresponding probability measure of the model is then defined by 
\begin{equation}
	\pi_{\text{Potts}}(\sigma) = \frac{1}{\scrZ_{\text{Potts}}} e^{\beta H(\sigma)}, \quad  \text{for} \ \sigma \in \Sigma,
\end{equation}
where $\scrZ_{\text{Potts}}$ is the partition function of the Potts model
\begin{equation}
	\scrZ_{\text{Potts}} = \sum_{\sigma\in \Sigma} \prod_{\langle ij \rangle \in E} \exp(\beta q\delta_{\sigma_i, \sigma_j}).
\end{equation}

While $q$ is assumed to be an integer greater than 1 in the above definition, 
Fortuin and Kasteleyn \cite{Fortuin1972} pointed out that it can be extended to any real number by reformulating the Potts model into another model, 
called the random-cluster model. Here we derive this result following the approach by Edwards and Sokal\cite{Edwards1988}.
They introduced the joint probability measure on the direct product of the spin configuration space $\Sigma$ and the bond configuration
space $\Omega = \{0, 1\}^{|E|}$:
\begin{equation}\label{eq:joint_measure}
	\mu(\sigma, \omega)	= \frac{1}{\scrZ_{\text{joint}}} \prod_{e=\langle ij \rangle \in E} \left[(1-p)\delta_{\omega_e, 0} + p\delta_{\omega_e, 1}\delta_{\sigma_i, \sigma_j}\right],
	\quad \text{for}\ (\sigma, \omega) \in \Sigma\times\Omega,
\end{equation}
where $\scrZ_{\text{joint}}$ is the partition function for the joint distribution
\begin{equation}
	\scrZ_{\text{joint}} = \sum_{\sigma\in \Sigma, \omega \in \Omega} \prod_{e=\langle ij \rangle \in E} \left[(1-p)\delta_{\omega_e, 0} + p\delta_{\omega_e, 1}\delta_{\sigma_i, \sigma_j}\right].
\end{equation}

The probability measure of the Potts model can be recovered by calculating the marginal distribution of $\sigma$ and setting $p = 1-e^{-\beta q}$:
\begin{equation}
	\begin{aligned}
	\mu_{\text{margin}}(\sigma) &= \sum_{\omega \in \Omega} \mu(\sigma, \omega) = \frac{1}{\scrZ_{\text{joint}}} \prod_{e=\langle ij \rangle \in E}\left[\sum_{\omega_e = 0}^1 \left((1-p)\delta_{\omega_e, 0} + p\delta_{\omega_e, 1}\delta_{\sigma_i, \sigma_j}\right)\right]\\
	&= \frac{1}{\scrZ_{\text{joint}}} \prod_{e=\langle ij \rangle \in E}\left[(1-p) + p\delta_{\sigma_i, \sigma_j}\right]\\
	&= \frac{1}{\scrZ_{\text{joint}}} \prod_{e=\langle ij \rangle \in E}e^{-\beta q} e^{\beta q\delta_{\sigma_i, \sigma_j}} 
	\propto e^{\beta q\sum_{\langle ij \rangle \in E}\delta_{\sigma_i, \sigma_j}} \propto \pi_{\text{Potts}}(\sigma).\\
	\end{aligned}
\end{equation}
Since both $\mu_{\text{margin}}(\sigma)$ and $\pi_{\text{Potts}}(\sigma)$ are normalized, we obtain $\mu_{\text{margin}}(\sigma) = \pi_{\text{Potts}}(\sigma)$,
and the partition functions are therefore related as
\begin{equation}\label{eq:Z_relation}
	\scrZ_{\text{joint}} = e^{-\beta q |E|} \scrZ_{\text{Potts}}.
\end{equation}
On the other hand, we can calculate the marginal measure of $\omega$ as
\begin{equation}\label{eq:RC_measure}
	\begin{aligned}
	\mu_{\text{margin}}(\omega) &= \sum_{\sigma \in \Sigma} \mu(\sigma, \omega)
	= \frac{1}{\scrZ_{\text{joint}}} q^{k(\omega)}\prod_{e=\langle ij \rangle \in E}\left[(1-p)\delta_{\omega_e, 0} + p\delta_{\omega_e, 1}\right],
	\end{aligned}
\end{equation}
where $k(\omega)$ is the number of connected components in $\omega$. In the above derivation, we use the fact that $\mu(\sigma, \omega) = 0$ if there exists an occupied edge $e = \langle ij \rangle$, i.e., $\omega_e = 1$,
such that its neighboring spins satisfy $\sigma_i \neq \sigma_j$. As a result, 
% in the summation $\sum_{\sigma\in \Sigma} \mu(\sigma, \omega)$,
only configurations having spins being in the same state for each connected component of $\omega$ gives non-zero contributions to $\sum_{\sigma\in \Sigma} \mu(\sigma, \omega)$ in Eq.~\eqref{eq:RC_measure}.
Moreover, because there are $q$ possible configurations (taking one of the $q$ values) for each connected component, the summation of $\sigma$ gives the extra factor $q^{k(\omega)}$. 
Equation~\eqref{eq:RC_measure} is exactly the probability measure $\pi_{\text{RC}}(\omega)$ for the random-cluster model with $\scrZ_{\text{RC}} = \scrZ_{\text{joint}}$.
Combining with Eq.~\eqref{eq:Z_relation}, we have $\scrZ_{\text{RC}} = e^{-\beta q|E|} \scrZ_{\text{Potts}}$. Hence, the random-cluster model
provides an equivalent representation of the $q$-state Potts model (with $p = 1-e^{-\beta q}$), which generalizes $q$ to real values. 

Besides the connection between the partition functions, a more interesting relation is between the two-point correlations in the Potts model
and the two-point connectivity in the random-cluster model\cite{Grimmett2006}. To state the result, we first rewrite the Hamiltonian Eq.~\eqref{eq:Potts_H} in
terms of the $(q-1)$-dimensional Potts spin vectors $u_i \in \mathbb{R}^{q-1}$, which satisfies 
\begin{equation}
	u_i \cdot u_j = q\delta_{\sigma_i, \sigma_j} - 1.
\end{equation}
The two-point correlation function in the Potts model is defined as 
\begin{equation}
	\langle u_x \cdot u_y \rangle_{\text{Potts}} = \frac{1}{\scrZ_{\text{Potts}}} \sum_{\{u\}}u_x \cdot u_y 
	\exp\left[\beta \sum_{\langle ij \rangle \in E} (u_i\cdot u_j  + 1) \right], \quad \forall x, y\in V.
\end{equation}
Then the result states 
\begin{equation}\label{eq:Potts_RC_corr}
	\frac{1}{q-1}\langle u_x \cdot u_y \rangle_{\text{Potts}} = \mathbb{P}_{\text{RC}}[x \leftrightarrow y],
\end{equation}
where $\mathbb{P}_{\text{RC}}[x\leftrightarrow y]$ is the connectivity probability between sites $x$ and $y$, i.e., the
probability that $x$ and $y$ are in the same components. Using Eq.~\eqref{eq:Potts_RC_corr}, information about the percolation
of the random-cluster model can be understood from the perspective of the long-range ordering of the Potts model.
The equivalence between the two systems has facilitated the study of both models, and we refer to Ref.~\inlinecite{Grimmett2006} for a complete discussion.

Now let's consider the limit of $q\to 0$ of Eq.~\eqref{eq:RC_measure}. The measure $\pi_{\text{RC}}$ can be rewritten as
\begin{equation}
	\begin{aligned}
	\pi_{\text{RC}}(\omega) &= (1-p)^{|E|}(\frac{p}{1-p})^{\eta(\omega)}q^{k(\omega)}\Bigg/\left[(1-p)^{|E|}\sum_{\omega \in \Omega}(\frac{p}{1-p})^{\eta(\omega)}q^{k(\omega)}\right]\\
	&= (\frac{p}{1-p})^{\eta(\omega)}q^{k(\omega)}\Bigg/\left[\sum_{\omega \in \Omega}(\frac{p}{1-p})^{\eta(\omega)}q^{k(\omega)}\right],\\
	\end{aligned}
\end{equation}
where $\eta(\omega)$ denotes the number of occupied edges. We further require that $p$ approaches 0 at the same rate as does $q$, i.e., $p = w q \to 0$, which simplifies $\pi_{\text{RC}}(\omega)$ to
\begin{equation}
	\pi_{\text{RC}}(\omega) = w^{\eta(\omega)}q^{\eta(\omega) + k(\omega)}\Bigg/\left[\sum_{\omega \in \Omega}w^{\eta(\omega)}q^{\eta(\omega) + k(\omega)}\right], \quad q \to 0.
\end{equation}
Since we have $\eta(\omega) + k(\omega) \geq |V|$, it follows that $\pi_{\text{RC}}(\omega)$ is only non-zero
when the equality holds, which selects out the spanning forests:
\begin{equation}
	\pi_{\text{RC}}(\omega) =
	\begin{cases}
	 w^{\eta(\omega)}/\sum_{\omega \in \scrF}w^{\eta(\omega)}, &\text{if}\ \omega \in \scrF, \\
	 0 &\text{otherwise}.
	\end{cases}
\end{equation}
Therefore, the model of spanning forests can be viewed as the special case of the Pott/random-cluster model with 
$q \to 0$ while keeping the ratio $w = p/q$ finite. 

\section{Grassmann variable}
To explore the critical behavior of this geometrical model, it is more convenient to work on its fermionic representation \cite{Caracciolo2004}, 
which allows a field-theoretic approach. For completeness, we shall first introduce the Grassman variable, which is the central object
of fermionic theories. 

A Grassmann algebra describes anticommuting objects, which is defined by a set of generators $\{\p_1, \dots,\p_n\}$ satisfying
\begin{equation}
	\{\psi_i, \psi_j\} = \p_i \p_j + \p_j \p_i = 0.
\end{equation}	
The anticommutativity of generators implies that they are nilpotent, i.e., $\psi_i^2 = 0$. Correspondingly, elements of the
Grassman algebra can be expressed as a polynomial of the first order in generators
\begin{equation}
    f(\psi_1, \cdots, \psi_n) = c_0 + \sum_{k=1}^{n}\sum_{i_1,\cdots,i_k = 1}^{n} c_{i_1,\cdots,i_k} \psi_{i_1}\cdots\psi_{i_k},
\end{equation}
where coefficients $c_{i_1, \cdots, i_k}$ are complex, and we assume $i_1 < \cdots < i_k$. Note that the ordering is important since interchanging generators gives an extra sign. 
The left derivative $\partial/\partial \psi_i$ of the generator $\psi_j$ is defined by
\begin{equation}
	\frac{\partial}{\partial \p_i} \p_j = \delta_{i, j}.
\end{equation}
By requiring it to be a linear operator as the ordinary case, the operator extends to any element of the algebra. For example,
for a function $f(\psi_1, \psi_2)$, its (left) derivative with respect to $\p_2$ is
\begin{equation}
	\begin{aligned}
	\frac{\partial}{\partial \p_2} f(\p_1, \p_2) &= \frac{\partial}{\partial \p_2} (c_0 + c_1 \p_1 + c_2 \p_2 + c_{12} \p_1 \p_2) \\
	&= c_2 - c_{12}\frac{\partial}{\partial\p_2}(\p_2 \p_1) = c_2 - c_{12}\p_1.
	\end{aligned}
\end{equation}
The differentiations also satisfy the anticommuatation relations $\{\partial_{\p_i}, \partial_{\p_j}\} = 0$.

For Grassman variables, the integration is identified with the derivatives
\begin{equation}\label{eq:integration}
	\int d\psi_i f(\p) = \frac{\partial}{\partial \psi_i} f(\p), \quad
	\int d\psi_i d\p_j f(\p) = \frac{\partial}{\partial \psi_i} \frac{\partial}{\partial \psi_j}f(\p),\quad \cdots
\end{equation}
Notice that Eq.~\eqref{eq:integration} implies that the rule for changing Grassmann variables in the integral is different from the ordinary case. 
For $\psi_j^\prime = A_{ji}\psi_i$, we have
\begin{equation}
    \begin{aligned}
    I &= \int d\psi_1\cdots d\psi_n f(\psi_1,\cdots,\psi_n) = \prod_{i=1}^n \frac{\partial}{\partial \psi_i} f(\psi_1,\cdots,\psi_n)\\
    &= \prod_{i=1}^n (\frac{\partial \psi^\prime_j}{\partial \psi_i}\frac{\partial}{\partial \psi_j^\prime} )f(\psi_1,\cdots,\psi_n)
    = \prod_{i=1}^n (A_{ji}\frac{\partial}{\partial \psi_j^\prime}) f(\psi_1,\cdots,\psi_n) \\
	&= \sum_{P\in S_n} (-1)^P A_{1 P(1)}\cdots A_{n P(n)} \int d\psi^\prime_1\cdots d\psi^\prime_n f(\psi_1,\cdots,\psi_n)\\
	&= \operatorname{det}A \int d\psi^\prime_1\cdots d\psi^\prime_n f(\psi_1,\cdots,\psi_n),
    \end{aligned}
\end{equation}
where for ordinary numbers, the factor $\operatorname{det} A$ is replaced by $(\operatorname{det} A)^{-1}$.

Now let's consider the Gaussian integrals of these Grassmann variables, which will be used to rewrite the
matrix tree theorem. In the following, we will consider the case of $2n$ generators $\{\psi_1, \dots, \psi_n, \bar{\psi}_1, \dots, \bar{\psi}_n\}$, where $\bar{\psi}_i$ 
and $\p_i$ are independent variables. The one-dimensional case is simple
\begin{equation}
    Z = \int d\p d\bar{\p}  e^{a\bar{\p} \p} = \int d\p d\bar{\p}(1+a\bar{\p}\p) = a,
\end{equation}
where $a$ is a complex number. The generalization to higher-dimensional case is
\begin{equation}
\begin{aligned}
    Z &= \int \left(\prod_{i=1}^n d\p_i d\bar{\p}_i\right)  e^{\sum_{i,j}\bar{\p}_i A_{ij} \p_j} = \operatorname{det}A \int \left(\prod_{i=1}^n d\p_i^\prime d\bar{\p}_i\right) 
    e^{\sum_i \bar{\p}_i\p_i^\prime}\\ 
    &= \operatorname{det}A \prod_{i=1}^n \int d\p_i^\prime d\bar{\p}_i(1+\bar{\p}_i\p^\prime_i) = \operatorname{det}A.
\end{aligned}
\end{equation}
In the derivation, we make a change of variables from $\p_i$ to $\p_i^\prime = \sum_j A_{ij} \p_j$, which introduces
the factor $\operatorname{det}A$ and factorizes the integral. We can also evaluate the expectation values of monomials with
respect to the Gaussian measure. By introducing the source terms and taking the corresponding derivatives,
one can prove the following identity
\begin{equation}\label{eq:Wick}
	\int \left(\prod_{i=1}^{n} d\psi_i d\bar{\psi}_i\right) \bar{\psi}_{i_{1}} \psi_{j_{1}} \bar{\psi}_{i_{2}} \psi_{j_{2}} \cdots \bar{\psi}_{i_{k}} \psi_{j_{k}} 
	e^{\sum_{i, j=1}^{n} \bar{\psi}_{i} A_{i j} \psi_{j}}=\epsilon(I|J) \operatorname{det} A(I|J),
\end{equation}
where $I = (i_1, \dots, i_k)$ and $J = (j_1, \dots, j_k)$. The sign $\epsilon(I|J) = (-1)^{\sum_{s=1}^k i_s - \sum_{s=1}^k j_s}$ arises from interchanging
Grassmann numbers, and $A(I|J)$ is the submatrix of $A$ by removing rows $I$ and columns $J$. In particular, for the case of $I = J$, we have $\epsilon(I|J) = 1$,
and $\operatorname{det} A(I|J)$ becomes a principal minor.

% Furthermore, consider introducing the source term $\{J_i, \bar{J}_i\}$ (Grassmann numbers),
% the generalized Gaussian integral is evaluated as
% \begin{equation}\label{eq:Gen_Gaussian}
% 	\begin{aligned}
% 	    Z(\bar{J}, J) &= \int \prod_{i=1}^n d\bar{\p}_i d\p_i \exp(-\sum_{ij} \bar{\p}_i A_{ij} \p_j + \sum_i \bar{J}_i \p_i + \sum_i \bar{\p}_i J_i)\\
% 	    &= \int \prod_{i=1}^n d\bar{\p}_i d\p_i \exp(-\sum_{ij} (\bar{\p}_i - \sum_k\bar{J}_{k}{A}^{-1}_{ki}) 
% 	    A_{ij} (\p_j - \sum_k A^{-1}_{jk} + \sum_i \bar{J}_i \p_i + \sum_i \bar{\p}_i J_i)
% 	\end{aligned}
% \end{equation}



\section{Matrix tree theorem and fermionic representation}
The derivation of the fermionic representation proceeds by making use of Kirchhoff's matrix tree theorem \cite{Kirchhoff1847} and its
generalizations \cite{Chaiken1982, Moon1994}, which relates the graph's Laplacian to the spanning trees or (rooted) spanning forest of the graph. 
For an undirected finite graph $G = (V, E)$ with vertices $V$ and edges $E$, we assign each edge $e \in E$ a weight $w_e$. The Laplacian matrix $L$ of 
the graph $G$ is then defined by
\begin{equation}
	L_{ij} =
	\begin{cases}
	-w_{ij}, & \langle ij \rangle \in E, \\
	\sum_{k(\neq i)} w_{ik} & i = j.
	\end{cases}
\end{equation}
The matrix tree theorem states that
\begin{equation}\label{eq:matrix_tree}
	\operatorname{det} L(i) = \sum_{T \in \scrT} \prod_{e\in T} w_e, \quad \forall i \in V,
\end{equation}
where $L(i)$ is the submatrix of $L$ by removing the $i$th row and column, and $\scrT$ is the collection of all spanning trees in $G$.
Notice that Eq.~\eqref{eq:matrix_tree} holds for arbitrary vertex $i \in V$. Furthermore, equation~\eqref{eq:matrix_tree} can be 
generalized to the case of a set of vertices $I = \{i_1, \dots, i_r\} \in V$, which has the form
\begin{equation}\label{eq:matrix_tree_gen}
	\operatorname{det}L(I) = \operatorname{det}L (i_1, \dots, i_r) = \sum_{F \in \scrF(i_1, \dots, i_r)} \prod_{e \in F} w_e, 
\end{equation}
where $L(I)$ now is the submatrix of $L$ by removing the same rows and columns contained in $I$, and $\scrF(i_1, \dots, i_r)$ is the collection
of rooted spanning forests in $G$ with roots $i_1, \dots, i_r$.
To connect with Grassmann variables, we express the principal minor $\operatorname{det} L(I)$ on the left hand side of Eq.~\eqref{eq:matrix_tree_gen} using Eq.~\eqref{eq:Wick}.
For each site $i \in V$, we introduce a pair of Grassmann variables $\psi_i, \bar{\psi}_i$ and rewrite Eq.~\eqref{eq:matrix_tree_gen} as
\begin{equation}\label{eq:matrix_tree_Grassmann}
	\int D(\p, \bar{\p})\left(\prod_{\alpha=1}^r \psi_{i_{\alpha}} \bar{\psi}_{i_{\alpha}}\right)e^{\bar{\p} L \p} 
	=\sum_{F \in \scrF(i_1, \dots, i_r)} \prod_{e \in F} w_e, 
\end{equation}
where we use $D(\p, \bar{\p})$ denote the integral measure $\prod_{i=1}^n d\p_i d\bar{\p}_i$ and 
represent the sum $\sum_{ij} \bar{\p}_i L_{ij} \p_j$ as the matrix product $\bar{\p} L \p$. 

For each connected subgraph $\G = (V_{\G}, E_{\G})$, we introduce the monomial 
\begin{equation}
	Q_{\G} = (\prod_{e\in E_\G}w_e)(\prod_{i\in V_{\G}}\bar{\p}_i \p_i),
\end{equation}
which commutes with the whole Grassmann algebra. Now let's consider a set of these subgraphs $\bm{\Gamma} = \{\G_1, \dots, \G_l\}$ and
evaluate the following integral
\begin{equation}
	\int D(\p, \bar{\p}) \left(\prod_{i=i}^l Q_{\G_i}\right) e^{\bar{\p} L \p}.
\end{equation}
We shall focus on the case when the subgraphs $\G_i$ are vertex-disjoint, otherwise the integral vanishes due to the nilpotency of the Grassmann generators.
Applying Eq.~\eqref{eq:matrix_tree_Grassmann}, we find the integral can be expressed as the sum of spanning forests with roots $V_{\bm{\Gamma}} = \bigcup_{i=1}^{l} V_{\G_i}$.
Moreover, the edges $E_{\bm{\Gamma}} = \bigcup_{i=1}^l E_{\G_i}$, which are absent in these forests if we do not include $\prod_{e\in E_{\G}} w_e$ in $Q_{\G}$, are added to
connect these rooted forests to form an object which we refer as the $\bm{\G}$-forest. More precisely, a $\bm{\G}$-forest corresponds to a spanning subgraph which recovers
to a spanning forest with roots $V_{\bm {\G}}$ after the deletion of $E_{\bm{\G}}$, and we use $\scrF_{\bm{\G}}$ denote the set of those $\bm{\G}$-forests. Therefore, we have
\begin{equation}
	\int D(\p, \bar{\p}) \left(\prod_{i=i}^l Q_{\G_i}\right) e^{\bar{\p} L \p} = \sum_{H \in \scrF_{\bm{\Gamma}}} \prod_{e\in H} w_e.
\end{equation} 
Summing over all possible sets $\bm{\G}$ (including the empty set), we obtain the following formula
\begin{equation}
	\int D(\psi, \bar{\psi}) \exp\left(\bar{\psi} L \psi+\sum_{\Gamma} t_{\Gamma} Q_{\Gamma}\right)=
	\sum_{\substack{\bm{\G}\  \text{vertex-}\\ \text{disjoint}} }\left(\prod_{\Gamma \in {\bm{\Gamma}}} t_{\Gamma}\right) \sum_{H \in \mathcal{F}_{\bm{\G}}} \prod_{e \in H} w_{e},
\end{equation}
where the summation of $\Gamma$ in the left hand side is over all possible connected subgraphs (not necessarily spanning).

Interchanging the summations over $\Gamma$ and $H$, that's, for each spanning subgraph $H$ with connected components ($H_1, \dots, H_l$), 
we first compute
\begin{equation}
	W(H_i) = \sum_{\G \prec H_i} t_{\G}
\end{equation}
for each component $H_i$, where $\G \prec H_i$ denotes the set of connected subgraphs $\G$ which relate to $H_i$.
Therefore, the integral is rewritten as
\begin{equation}
	\int D(\psi, \bar{\psi}) \exp\left(\bar{\psi} L \psi+\sum_{\Gamma} t_{\Gamma} Q_{\Gamma}\right)=
	\sum_{\substack{H\ \text{spanning-}\\ \text{subgraph}}} \left(\prod_{i=1}^l W(H_i)\right) \prod_{e \in H} w_{e}.
\end{equation}

Consider the special case in which $t_{\G} = t$ when $V_{\G}$ consists of a single vertex and $E_{\G}$ is empty, $t_{\G} = -t$ when $V_\G$
consists of two neighboring vertices and $E_{\G}$ contains the edge connecting them, and $t_{\G} = 0$ otherwise. Then $W(H_i)$ is non-zero only when
$H_i$ is a tree, and the integral recovers the partition function of the spanning forests:
\begin{equation}\label{eq:fermionic_repr}
	\begin{aligned}
	\int &D(\psi, \bar{\psi}) \exp \left[\bar{\psi} L \psi+t \sum_{i} \bar{\psi}_{i} \psi_{i} -t \sum_{\langle i j\rangle} w_{i j} \bar{\psi}_{i} \psi_{i} \bar{\psi}_{j} \psi_{j}\right]\\
	 &= \sum_{\substack{F \in \scrF \\ F = (F_1, \dots, F_l)}} \prod_{i=1}^l\left(t|V_{F_i}| - t|E_{F_i}|\right) \prod_{e\in F} w_e 
	 = t^{|V|} \sum_{F\in \scrF}  \prod_{e\in F} \frac{w_e}{t},
	\end{aligned}
\end{equation}
where in the last equality we use the identity $\#(\text{components}) = |V| - |E_{F}|$ for spanning forests. On the left side of Eq.~\eqref{eq:fermionic_repr}, 
we have a fermionic theory with a Gaussian term and a nearest-neighbor four-fermion interaction. The coupling $t$ is actually redundant in the above formulation. To see this, 
We rewrite the weight $w_e^\prime = w_e/t$ and rescale the field $\p \to \p^\prime =\sqrt{t} \p, \bar{\p} \to \bar{\p^\prime} = \sqrt{t}\bar{\p}$, which
gives the following formula
\begin{equation}
	\int D(\psi, \bar{\psi}) \exp \left[\bar{\psi} L \psi+ \sum_{i} \bar{\psi}_{i} \psi_{i} - \sum_{\langle i j\rangle} w_{i j} \bar{\psi}_{i} \psi_{i} \bar{\psi}_{j} \psi_{j}\right]\\
	 = \sum_{F\in \scrF}  \prod_{e\in F} w_e,
\end{equation}
where we further omit the superscript.

In Ref.~\inlinecite{Roland2021Hyperbolic}, it was noticed that Eq.~\eqref{eq:fermionic_repr} describes a fermionic version of the hyperbolic sigma model, which 
is referred to as the $\mathbb{H}^{0|2}$ model for short. In the $\mathbb{H}^{0|2}$ model, in addition to the Grassmann pair $\psi_i, \bar{\psi}_i$, we introduce
the bosonic field
\begin{equation}
	z_i = \sqrt{1 - 2\b{\psi}_i\psi_i} = 1-\bar{\p}_i\p_i,
\end{equation}
which is an even element in the Grassmann algebra. The inner product between triplets $u_i = (\p_i, \bar{\p}_i, z_i)$ and 
$u_j = (\p_i, \bar{\p}_i, z_i)$ are defined as
\begin{equation}
	u_i \cdot u_j = -\bar{\p_i}\p_j - \bar{\p_j}\p_i - z_i z_j = -1 - \bar{\p_i}\p_j - \bar{\p_j}\p_i + \bar{\p}_i\p_i + \bar{\p}_j\p_j
	-\bar{\p}_i\p_i\bar{\p}_j\p_j.
\end{equation}
Note that $u_i \cdot u_i = -1$, for which $u_i$ is interpreted as a supervector taking values in a hyperbolic space. 
The action of the model is defined by $S = \frac{1}{2}\sum_{\langle ij \rangle} w_{ij}(u_i - u_j)^2$, identical to the non-linear sigma model.
The expectation value of observable $F$ is written as
\begin{equation}
	\langle F \rangle = \frac{1}{Z} \int \left(\prod_{i}d\p_i d\bar{\p}_i \frac{1}{z_i}\right) F e^S,
	\quad Z = \int \left(\prod_{i}d\p_i d\bar{\p}_i \frac{1}{z_i}\right) e^S,
\end{equation}
where the factor $1/z_i$ arises from the Haar measure over the hyperbolic space. Since $1/z_i = 1/(1-\bar{\p_i}\p_i) = 1 + \bar{\p_i}\p_i = e^{\bar{\p_i}\p_i}$,
the expectation value of $F$ in the $\mathbb{H}^{0|2}$ sigma model is exactly the same as the corresponding expectation value in the spanning-forest model.
In particular, it's shown that the two-point correlation function of the $\mathbb{H}^{0|2}$ model is identical to the connection probability 
in the spanning forests
\begin{equation}\label{eq:H02_SPF_corr}
-\langle u_x \cdot u_y \rangle_{\mathbb{H}^{0|2}} = \mathbb{P}_{\text{SPF}}[x \leftrightarrow y], \quad \forall x, y \in V.
\end{equation}
Equation~\eqref{eq:H02_SPF_corr} is in close analogy with Eq.~\eqref{eq:Potts_RC_corr} for the Potts model. Roughly speaking,
the hyperbolic sigma model can be regarded as the $0$-state Potts model, which has a continuous symmetry group $\text{OSp}(1|2)$
instead of the discrete symmetry group $S_q$ for the $q\geq2$ Potts model. The relation between the 

Using Eq.~\eqref{eq:H02_SPF_corr}, Roland et al. proved that the percolation of the spanning forests only happens for $d\geq 3$.
Moreover, they obtained that, for a sequence of tori $\Lambda_N = \mathbb{Z}^d/L^N\mathbb{Z}^d$, the connectivity probability
in the supercritical phase can be written as
\begin{equation}\label{eq:SPF_conn}
	\mathbb{P}_{\text{SPF}}[0\leftrightarrow x] = \zeta_{d}(w)+\frac{c(w)}{w|x|^{d-2}}+O\left(\frac{1}{w|x|^{d-2+\kappa}}\right)+O\left(\frac{1}{w L^{\kappa N}}\right),
\end{equation}
where $\zeta_{d}(\beta)=1-O(1 / \beta), c(\beta)=c+O(1 / \beta)$ and $\kappa > 0$. It's interesting to note that the supercritical
phase of the spanning forests behaves differently from the supercritical phase of the bond percolation. For the spanning forests,
the model behaves like a critical system even in the supercritical phase in the sense that the subleading term in Eq.~\eqref{eq:SPF_conn}
decays as the correlation of a massless free field. This phenomenon is already discovered in the case of complete graphs. 
In Ref.~\inlinecite{Luczak1992,Martin2018}, the authors have shown that, for the complete graph, there is an unbounded number of components whose sizes are of order $|V|^{2/3}$.
Interestingly, the same scaling law $|V|^{2/3}$ appears at the critical point on the complete graph. In the following, 
we will explore this phenomenon in $d=3$ by focusing on the finite-size scaling behaviors of the non-giant components of the model, which
complements the result from Eq.~\eqref{eq:SPF_conn}.
