\chapter{Results}\label{chap:results}

\section{Phase transition}\label{sec:phase_transition}
We first study the percolation transition of the spanning forests on the complete graph (CG) of finite volume $V$ and
a three-dimensional cubic lattice with periodic boundary conditions. During the simulation, after every Monte Carlo sweep, we sample 
\begin{itemize}
	\item the size of the largest component $\scrC_1$;
	\item the graph diameter of the largest component $\scrD_1$;
	\item the cluster size moments $\scrS_k = \sum_i (\scrC_i)^k$ for $k=2, 4$;
	\item the cluster number $\scrN(s)$, defined as the number of clusters whose sizes are within $[s, s + \Delta s]$ 
	with an appropriately chosen interval size $\Delta s$.

\end{itemize}
and compute the following quantities:
\begin{itemize}
	\item the averaged size of the largest component: $C_1 = \langle \scrC_1 \rangle$;
	\item the averaged graph diameter of the largest component: $D_1 = \langle \scrD_1 \rangle$;
	% \item the dimensionless ratio $Q_{\rm c1} = \langle \scrC_1^2 \rangle\big/\langle \scrC_1 \rangle^2$;
	\item the dimensionless ratio\cite{Qian2005} $Q_S = \langle \scrS_2\rangle^2\big/\left(3\langle \scrS_2^2 \rangle - \langle \scrS_4\rangle\right)$;
	\item the cluster-size distribution $n(s)= \frac{1}{V\Delta s} \langle \scrN(s)\rangle$.
	% \item the derivative of $\ln Q_{\rm c1}$ with respect to $\ln w$, which is written as 
	% \begin{equation}
	% 	\frac{\partial \ln Q_{\rm c1}}{\partial \ln w}= 
	% 	\frac{\langle \scrN_b \scrC_1^2 \rangle}{\langle \scrC_1^2\rangle} + \langle \scrN_b \rangle - \frac{2 \langle \scrN_b \scrC_1 \rangle}{\langle \scrC_1 \rangle}.
	% 	\end{equation}
\end{itemize}

The dimensionless quantity $Q_{S}$ is well suited for exploring the existence of a phase transition point.
Its finite-size scaling behavior near the critical point $w_c$ (if exists) is $Q_{S}(w, L) = \tilde{Q}_{S}((w-w_c)L^{1/\nu})$,
where $\tilde{Q}_{S}$ is a universal function and $L$ is the linear system size of the cubic lattice ($L$ is replaced by $V$ for CG). Therefore, one should find that curves of different system sizes intersect
at $\left(w_c, \tilde{Q}_{S}(0)\right)$ if there exists a second-order phase transition. This is clearly observed for the spanning-forest model,
as shown in Fig.~\ref{fig:wc}. The quantity in the thermodynamic limit tends to 1/3 in the subcritical phase and to 1 in the supercritical phase,
in between a phase transition is presented at $\tilde{w}_c \equiv w_c V \approx 1$ for CG and at $w_c \approx 0.44$ for the cubic lattice,
consistent with the previous results\cite{Luczak1992,Deng2007,Bedini2009,Roland2021Percolation}.

\begin{figure}[t]
	\centering
	\includegraphics[width=0.85\columnwidth]{figures/Qs_wc.pdf}
	\caption{The dimensionless quantity $Q_S$ versus the weight $w$ for (a) CG and (b) the cubic lattice. 
	In Fig.~\ref{fig:wc}(a), we use $\tilde{w} = wV$ as the horizontal coordinate. An intersection is observed at $\tilde{w}_c \approx 1$. 
	For the cubic lattice, an excellent intersection is observed at $w_c \approx 0.44$. }
	\label{fig:wc}
\end{figure}

One can further accurately estimate the phase transition point $w_c$ by applying finite-size analysis to the data, but because our focus is the 
scaling behaviors of the clusters instead of determining $w_c$ with high-precision, we will use 
the theoretical result by Bedini et al. \cite{Bedini2009}, which gives $\tilde{w}_c = 1$ for CG, and
the numerical estimate by Deng et al.\cite{Deng2007}, which gives $w_c = 0.43365$ for 3D. We perform extensive simulations at these
two critical points with system size up to $V = 2^{21}$ for CG and to $L=128$ for 3D and determine various critical exponents from the quantities described above. 

\subsection{Complete Graph}
In this section, we investigate the critical behaviors of the spanning-forest on CG.
We first estimate the fractal dimension $d_{1 \textrm{f}}$ of the largest cluster at criticality on CG, which is defined as $C_1(w_c, V) \sim V^{d_{1\textrm{f}}}$.
We perform the least-squares fits to $C_1$ with the following ansatz
\begin{equation}\label{eq:C1_ansatz_CG}
    C(w_c, V) = V^{d_{\textrm{f}}}(a_0 + a_1V^{\omega_1} +a_2 V^{\omega_2}) + c_0,
\end{equation}
where terms $a_1, a_2,$ and $c_0$ are added to account for finite-size corrections, and we set $\omega_1 = -\frac{1}{3}$ and $\omega_2 = -\frac{2}{3}$ in the fit of $C_1$.
As a precaution against correction-to-scaling terms that we miss including in the fitting ansatz, we impose a lower cutoff $V \geq V_m$ 
on the data points admitted in the fits. We systematically study the effect on the residuals (denoted by chi$^2$) by increasing $V_m$. In general, 
the preferred fit for any given ansatz corresponds to the smallest $V_m$ for which the goodness of the fit is reasonable and for which subsequent 
increases in $V_m$ do not cause the chi$^2$ value to drop by more than one unit per degree of freedom (DF). In practice, by `reasonable' 
we mean that chi$^2$/DF $\approx$ 1. The systematic error is obtained by comparing estimates from various reasonable fitting ansatz.
Table~\ref{tab:C1_wc_CG} reports our results, where parameters fixed to 0 are denoted by `--'. We tried several ansatzes by
leaving various subsets of the fitting parameters free and obtain the final estimate $\dof = 0.6664(9)$, consistent with the theoretical 
prediction $\dof = \frac{2}{3}$.

\begin{table}[tb]
\centering
\vspace{3ex}
\caption{Fitting results of $\dof$ for the spanning forests on CG at $\tilde{w}_c = 1$ with ansatz Eq.~\eqref{eq:C1_ansatz_CG}.}
\label{tab:C1_wc_CG} 
\begin{tabular}{|lllllc|}
\hline 
$V_{m}$   &\multicolumn{1}{c}{$\dof$}   &\multicolumn{1}{c}{$a_0$}    
&\multicolumn{1}{c}{$a_1$} &\multicolumn{1}{c}{$c_0$}
& chi$^2/{\rm DF}$     \\
\hline 
$8^3$     &0.666\,8(4)    &0.751(5)   &0.11(5)    &0.1(3)     &9.2/7\\ 
$16^{3}$    &0.666\,4(9)    &0.76(1)    &0.0(2)     &1(1)       &9.0/6\\ 

$16^{3}$    &0.666\,9(3) 	&0.750(3)  	&0.12(2)   	&\multicolumn{1}{c}{-}	&9.3/7\\ 
% 24    &0.6667(5) 	&0.752(5)  	&0.10(5)   	&\multicolumn{1}{c}{-}	&9.1/6\\ 
$32^{3}$    &0.666\,6(7) 	&0.754(8)  	&0.07(9)   	&\multicolumn{1}{c}{-}	&9.0/5\\ 
% 48    &0.666(2)  	&0.77(2)   	&-0.1(3)   	&\multicolumn{1}{c}{-}	&8.5/4\\ 

$16^{3}$    &0.666\,3(2) 	&0.757(2)  	&\multicolumn{1}{c}{-}     	&1.0(2)    	&9.0/7\\ 
$32^{3}$    &0.666\,3(4) 	&0.757(4)  	&\multicolumn{1}{c}{-}     	&1(1)      	&8.8/5\\ 
% $2^{18}$    &0.664(2)  	&0.78(2)   	&\multicolumn{1}{c}{-}     	&-18(14)	&3.7/2\\ 
\hline 
\end{tabular} 
\vspace{3ex}
\end{table} 

Another quantity that characterizes the percolation transition of the spanning forests is the cluster-size distribution $n(s, V)$, which has the following
scaling form at criticality:
\begin{equation}\label{eq:ns_CG}
	n(s, V) = s^{-\tau} \tilde{n}(s/V^{d_{1\textrm{f}}}),
\end{equation}
where $\tau = 1+ 1/\dof = 5/2$ is the Fisher exponent and $\tilde{n}$ is a universal function. 
In Fig.~\ref{fig:ns_wc}(a), we plot the data of $n(s, V)$ on a log-log scale. For small cluster sizes $s \ll V^{\dof}$, the distribution exhibits a clear power-law behavior, which
is expected from Eq.~\eqref{eq:ns_CG} as the function $\tilde{n}(s V^{-d_{1\textrm{f}}})$ is close to the constant $\tilde{n}(0)$ in this
circumstance. When $s$ is comparable to $V^{\dof}$, the finite-size cutoff takes effect, which leads to the deviation from the power-law decay.
Furthermore, the inset of Fig.~\ref{fig:ns_wc}(a) shows the scaled distribution $s^{\tau}n(s, V)$ versus $s/V^{\dof}$, 
The good data collapse of different system sizes confirms the scaling form~\eqref{eq:ns_CG}.

\begin{figure}[h]
	\vspace{3ex}
	\centering
	\includegraphics[width=0.85\columnwidth]{figures/ns_wc.pdf}
	\caption{The log-log plot of the cluster-size distribution at criticality for (a) CG and (b) the cubic lattice. 
	The slopes of the black dashed lines in the two figures are $-5/2$ and $-2.161$, respectively. 
	The insets show the scaled distribution $s^{\tau} n(s, V)$ ($s^{\tau} n(s, L)$) versus $s/V^{\dof}$ ($s\big/L^{\dof}$).}
	\label{fig:ns_wc}
\end{figure}

\subsection{Cubic lattice}
The analysis of the model in 3D proceeds similarly. In the following, we use the linear system size $L$ instead of the system volume $V$
to characterize the scaling behaviors of the observables. The fractal dimension of the largest component at $w_c$ 
is defined as $C_1(w_c, L) \sim L^{d_{1\textrm{f}}}$, and we propose the following ansatz 
\begin{equation}\label{eq:C1_ansatz}
    C(w_c, L) = L^{d_{\textrm{f}}}(a_0 + a_1L^{\omega_1} +a_2 L^{\omega_2}) + c_0
\end{equation}
to fit the data, where terms $a_1, a_2$ and $c_0$ are added to account for finite-size corrections. For convenience, we fix $\omega_1 = -1$ and $\omega_2 = -2$ in the fit 
of $C_1$. 
Table~\ref{tab:C1_wc_3D} presents our results. We first leave all the parameters free and find the central values of
$a_2$ and $c_0$ are consistent with 0, which therefore are not reported in Table~\ref{tab:C1_wc_3D}. We then perform another two fits
with $a_2 = c_0 = 0$, which gives the final estimate of $d_{1\textrm{f}} = 2.5838(6)$.

\begin{table}[h]
\centering
 \vspace{3ex}
\caption{Fitting results of $d_{1\textrm{f}}$ for the spanning forests on the cubic lattice at $w_c = 0.43365$ with ansatz Eq.~\eqref{eq:C1_ansatz}.}
\label{tab:C1_wc_3D} 
\begin{tabular}{|lcccc|}
\hline 
$L_{m}$   &$d_f$   &$a_0$   &$a_1$    & chi$^2/{\rm DF}$     \\
\hline 
8     &2.5836(3)    &0.855(1)   &-0.007(5)  &2.0/5\\ 
16    &2.5839(5)    &0.853(2)   &0.01(2)    &1.3/4\\ 
8     &2.5839(1)    &0.8535(3)  &-          &3.5/6\\ 
16    &2.5837(1)    &0.8539(5)  &-          &1.4/5\\ 
32    &2.5838(2)    &0.8535(8)  &-          &0.7/3\\ 
\hline 
\end{tabular} 
 \vspace{3ex}
\end{table} 

% We also determine the shortest-path exponent $d_{1\min}$ of the largest cluster, which describes
% how the averaged graph distance $\langle l \rangle$ in the cluster relates to the Euclidean distance $r$, 
% namely $\langle l \rangle \sim r^{d_{1\min}}$. In practice, we measure the averaged graph diameter $D_1$, which
% behaves as $D_{1} \sim L^{d_{1\min}}$ at $w_c$ since the linear size of the largest cluster is cut off by $L$ at criticality.
% We use the same ansatz Eq.~\eqref{eq:C1_ansatz} to fit $D_{1}$, with the exponent $d_{1\textrm{f}}$ replaced by $d_{1\min}$.
% Again, we set $\omega_1 = -1, \omega_2 = -2$. The fitting results are presented in Table~\ref{tab:D1_wc_3D}.
% By comparing estimates from various ansatz, we finally obtain $d_{1\min} = 1.481(3)$.

% \begin{table}[h]
% \centering
%  \vspace{3ex}
% \caption{Fitting results of $d_{1 \min}$ for the spanning forests on the cubic lattice at $w_c = 0.43365$ with ansatz Eq.~\eqref{eq:C1_ansatz}.}
% \label{tab:D1_wc_3D} 
% \begin{tabular}{|llccccc|}
% \hline 
% $L_{m}$   &\multicolumn{1}{c}{$d_{1\min}$}   &$a_0$   &$a_1$  &$a_2$ & $c_0$ &chi$^2/{\rm DF}$     \\
% \hline 
% 8     &1.4797(5)    &2.274(5)   &-0.66(6)   &-1.9(3)   &- &5.9/4\\ 
% 16    &1.4817(10)   &2.25(1)    &-0.3(2)    &-5(1)     &- &0.4/3\\ 
% 8     &1.4806(6) 	&2.263(7)  	&-0.3(1)   	&-	&-1.5(2)   	&4.4/4\\ 
% 16    &1.483(1)  	&2.24(2)   	&0.4(4)    	&-	&-3.2(9)   	&0.5/3\\ 
% 24    &1.4798(5)    &2.274(6)   &-0.73(6)   &-         &- &1.4/3\\ 
% 32    &1.4805(8)    &2.266(9)   &-0.6(1)    &-         &- &0.1/2\\ 
% \hline 
% \end{tabular} 
%  \vspace{5ex}
% \end{table} 

We then examine the cluster-size distribution $n(s, L)$ at $w_c$, which now can be written as $n(s, L) = s^{-\tau} \tilde{n}(s L^{-d_{1\textrm{f}}})$ with
the Fisher exponent $\tau = 1 + 1/\dof$
Figure.~\ref{fig:ns_wc}(b) shows the distribution on a log-log scale, where the exponent $2.616$ is obtained by plugging our estimate $\dof=2.5838(6)$
into the relation $\tau = 1+1/\dof$. The excellent data collapse in the inset confirms both the correctness of our estimate of $\dof$ and the scaling form of $n(s, L)$.

\section{Supercritical phase}
In this section, we explore the critical behaviors of the spanning forests in its supercritical phase for both CG and the cubic lattice.
Besides the quantities mentioned in Sec.~\ref{sec:phase_transition}, we further measure the size of the second-largest cluster $\scrC_2$ and 
compute its average $C_2$.
% \begin{itemize}
	% \item the size of the second-largest component $\scrC_2$;
	% \item the graph diameter of the second-largest component $\scrD_2$;
	% TODO Unwrapped Radius
% \end{itemize}
% and compute the following quantities:
% \begin{itemize}
% 	\item the averaged size of the second-largest component: $C_2 = \langle \scrC_2 \rangle$;
% 	\item the averaged graph diameter of the second-largest component: $D_2 = \langle \scrD_2 \rangle$.
% \end{itemize}
In addition, using our recorded samples, we generate the probability density functions for $\scrC_1$ and $\scrC_2$.

\subsection{Complete Graph}


We simulate the model on CG at $\tilde{w}_1 = 2 > \tilde{w}_c$. As pointed out in Ref.~\inlinecite{Luczak1992,Martin2018}, apart from
a macroscopic cluster of the scale $O(V)$, there are an unbounded number of clusters whose sizes are of the order $(V^{2/3})$ in the
supercritical phase. To confirm the statement, we estimate the fractal dimension $\dtf$ of the second-largest cluster by fitting the data of $C_2$ using ansatz~\eqref{eq:C1_ansatz_CG}.
We set $\omega_1 = -1/3$ and $\omega_2 = -1/2$ in the fit of $C_2$. The results are summarized in Table~\ref{tab:C2_w1_COM}, and we obtain $\dtf = 0.660(6)$, 
consistent with the prediction $2/3$. Figure.~\ref{fig:C1_C2_CG} plots the cluster sizes $C_1(\tilde{w}_c, V)$ and $C_2(\tilde{w}_1, V)$ versus the system volume
$V$ on a log-log scale. The data points of $C_1(\tilde{w}_c, L)$ and $C_2(\tilde{w}_1, L)$ are almost indistinguishable for large $V$, and both quantities
can be fit by the power-law behavior with exponent $\frac{2}{3}$.

\begin{figure}[ht]
	\centering
	\vspace{3ex}
	\includegraphics[width=0.5\columnwidth]{figures/C1_C2_CG.pdf}
	\caption{The scaling behaviors of $C_1(\tilde{w}_c, V)$ and $C_2(\tilde{w}_1, V)$ on CG, plotted on a log-log scale. The dashed line 
	represents $C = a_0 V^{d_{\text{f}}}$ with $d_\text{f} = 2/3$.}
	\label{fig:C1_C2_CG}
\end{figure}

\begin{table}[htb]
\centering
\vspace{3ex}
\caption{Fitting results of $\dtf$ for the spanning forests on CG at $\tilde{w}_1 = 2$ with ansatz Eq.~\eqref{eq:C1_ansatz_CG}.}
\label{tab:C2_w1_COM} 
\begin{tabular}{|llllllc|}
\hline 
$V_{m}$   &\multicolumn{1}{c}{$\dtf$}   &\multicolumn{1}{c}{$a_0$}    
&\multicolumn{1}{c}{$a_1$}  &\multicolumn{1}{c}{$a_2$} &\multicolumn{1}{c}{$c_0$}
& chi$^2/{\rm DF}$     \\
\hline 
$24^3$    &0.6562(10)   &0.88(1)    &-1.5(1)    &\multicolumn{1}{c}{-} &\multicolumn{1}{c}{-}      &2.9/5\\ 
$48^3$    &0.660(3)     &0.83(5)    &-0.9(6)    &\multicolumn{1}{c}{-} &\multicolumn{1}{c}{-}      &1.5/3\\ 

$24^3$    &0.6619(6)    &0.798(6)   &\multicolumn{1}{c}{-} &\multicolumn{1}{c}{-}      &-15(1)     &2.0/5\\ 
$48^3$    &0.662(2)     &0.80(2)    &\multicolumn{1}{c}{-} &\multicolumn{1}{c}{-}      &-15(10)    &1.5/3\\ 

$16^3$    &0.658(2)     &0.85(3)    &-0.9(4)    &\multicolumn{1}{c}{-}     &-6(3)      &2.0/5\\ 
$24^3$    &0.660(4)     &0.82(5)    &-0.5(9)    &\multicolumn{1}{c}{-}     &-10(10)    &1.7/4\\ 

$16^3$    &0.659(2)     &0.83(3)    &-0.2(7)     &-4(2)       &\multicolumn{1}{c}{-}      &1.9/5\\ 
$24^3$    &0.661(5)     &0.80(7)    &0(2)        &-6(5)       &\multicolumn{1}{c}{-}      &1.7/4\\ 
\hline 
\end{tabular} 
\vspace{3ex}
\end{table} 


\begin{figure}[b]
	\centering
	\includegraphics[width=0.85\columnwidth]{figures/Prob_w1_CG.pdf}
	\caption{The semi-log plot of the probability density functions $f_{X_1}(x)$ and $f_{X_2}(x)$ at $\tilde{w}_1$ for various system sizes,
	where $X_1 \equiv (\scrC_1 - C_1)/V^{2/3}$ and $X_2 \equiv \scrC_2/V^{2/3}$.}
	\label{fig:prob_w1_CG}
\end{figure}

To characterize the behaviors of $\scrC_1$ and $\scrC_2$ more precisely, we plot the probability density functions
of variables $X_1 \equiv \frac{\scrC_1 - C_1}{V^{2/3}}$ and $X_2 \equiv \frac{\scrC_2}{V^{2/3}}$ for various system sizes using a semi-log scale,
as shown in Fig.~\ref{fig:prob_w1_CG}. For the variable $X_1$, we find that the data collapse to a single curve when $X_1$ is approximately larger
than -2. In this region, the fluctuations of $\scrC_1$ are of the order of $O(V^{2/3})$, in agreement with the result for the infinite-volume limit from Ref.~\inlinecite{Luczak1992}.
However, for finite systems, there exists a region with a vanishing probability in which the data fail to collapse. Instead, a preliminary plot of the 
probability density function $f_{X_1^\prime}(x)$ multiplied by $V^{1/2}$ for the variable $X_1^\prime = \frac{\scrC_1 - C_1}{V}$ exhibits a good data collapse for this region, 
as illustrated in Fig.~\ref{fig:prob_w1_CG_tail}(a). This implies that there exists a sector in the configuration space of the spanning forests which vanishes
asymptotically as $V^{-1/2}$, and the fluctuations of $\scrC_1$ in this sector are of the order of $O(V)$. 
For the second-largest cluster size $\scrC_2$, we also observe a similar phenomenon, as shown in Fig~\ref{fig:prob_w1_CG}(b) 
and Fig.~\ref{fig:prob_w1_CG_tail}(b), where $X_2^\prime$ is defined by $X_2^\prime = \frac{\scrC_2 - C_2}{V}$. 
Therefore, for a CG with finite volume $V$, there seems to exist two sectors in the configuration space of the spanning forests at the supercritical phase, which exhibit different finite-size scaling behaviors. 
Sector~\Romannum{1} is responsible for the critical behaviors of the model in the infinite-volume limit, in which $\scrC_2$ scales as $V^{2/3}$ and the fluctuations of 
$\scrC_1$ are of the order of $O(V^{2/3})$. On the other hand, sector~\Romannum{2} decays asymptotically with a rate $V^{-1/2}$, in which $\scrC_2$ scales
as $V$ and the fluctuations of $\scrC_1$ are of the order of $O(V)$.

\begin{figure}[htb]
	\centering
	\includegraphics[width=0.85\columnwidth]{figures/Prob_tail_w1_CG.pdf}
	\caption{The semi-log plot of the probability density functions $f_{X_1^\prime}(x)$ and $f_{X_2^\prime}(x)$ multiplied by $V^{1/2}$,
	where $X_1^\prime \equiv (\scrC_1 - C_1)/V$ and $X_2^\prime \equiv (\scrC_2-C_2)/V$.}
	\label{fig:prob_w1_CG_tail}
\end{figure}

We further study the cluster-size distribution in the two sectors. To suppress the finite-size corrections brought by
the crossover region between the two sectors, which is irrelevant in demonstrating the physics, we sample quantities conditioned on 
$\scrC_1 - C_1 > 0$ for the sector~\Romannum{1}, and sample quantities conditioned on $\scrC_1 - C_1 < -0.05V$ for the sector~\Romannum{2}.
Since we are interested in the non-giant clusters, we exclude $\scrC_1$ during the measurement and denote the resulting
cluster-size distribution as $n^\prime(s, V)$. Figure~\ref{fig:nsp_w1_CG} shows the distribution $n^\prime(s, V)$ for the two sectors on a log-log scale.
For sector~\Romannum{1}, we find that $n^{\prime}(s, V)\sim s^{-5/2}$ in the bulk region, and the distribution is well described by the formula $n^\prime(s, V) = s^{-5/2} \tilde{n}^\prime(s/V^{2/3})$, 
which is the same scaling form of $n(s, V)$ at criticality. For sector~\Romannum{2}, we also observe a power-law decay $n^\prime(s, V)\sim s^{-5/2}$ in the bulk
region. Moreover, there is another power-law behavior (indicated by the blue dashed line in Fig.~\ref{fig:nsp_w1_CG}(b)) with exponent $-2$.
This is probably due to the effect of $\scrC_2$ since its fractal dimension in sector~\Romannum{2} is $1$, which gives the Fisher exponent $\tau = 2$.

\begin{figure}[tb]
	\centering
	\includegraphics[width=0.85\columnwidth]{figures/nsp_w1_CG.pdf}
	\caption{The log-log plot of the cluster-size distribution $n^\prime(s, V)$ on CG sampled in (a) sector~\Romannum{1} and (b) sector~\Romannum{2}. The insets show
	the scaled distribution $s^\tau n(s, V)$ versus $s/V^{d_f}$, where the values of $\tau$ and $d_\text{f}$ are indicated in the legends.}
	\label{fig:nsp_w1_CG}
\end{figure}

\subsection{Cubic lattice}
We simulate the model on the cubic lattice with periodic boundary conditions at $w = 0.9  > w_c$. 
We first estimate the fractal dimension $\dtf$ of the second-largest cluster, defined as $C_2 \sim L^{\dtf}$, using ansatz~\eqref{eq:C1_ansatz}.
We set $\omega_1 = -1$ and $\omega_2 = -3/2$ in the fit of $C_2$. The results are reported in Table~\ref{tab:C2_w1_3D}, from which we
obtain $\dtf = 2.30(3)$. Note that the value is different from the fractal dimension $\dof = 2.5838(6)$ at criticality,
as shown in Fig.~\ref{fig:C1_C2_3D}, where we plot $C_1(w_c, L)$ and $C_2(w_1, L)$ versus the linear system size $L$. 

\begin{table}[h]
\centering
\vspace{3ex}
\caption{Fitting results of $\dtf$ at $w_2 = 0.9$ for the spanning forests on the cubic lattice with ansatz Eq.~\eqref{eq:C1_ansatz}.}
\label{tab:C2_w1_3D} 
\begin{tabular}{|llllllc|}
\hline 
$L_{\rm min}$   &\multicolumn{1}{c}{$\dtf$}   &\multicolumn{1}{c}{$a_0$}    
&\multicolumn{1}{c}{$a_1$}  &\multicolumn{1}{c}{$a_2$} &\multicolumn{1}{c}{$c_0$}
& $\chi^2/{\rm DF}$     \\
\hline 
% 32    &2.303(3)     &0.474(8)   &-1.3(1)    &\multicolumn{1}{c}{-} &\multicolumn{1}{c}{-} &5.2/5\\ 
48    &2.294(8)     &0.50(2)    &-1.7(3)    &\multicolumn{1}{c}{-} &\multicolumn{1}{c}{-} &3.7/4\\ 
56    &2.28(1)      &0.54(4)    &-2.5(6)    &\multicolumn{1}{c}{-} &\multicolumn{1}{c}{-} &0.9/3\\ 

48    &2.321(4)     &0.426(8)   &\multicolumn{1}{c}{-} &\multicolumn{1}{c}{-}      &-80(15)    &5.8/4\\ 
56    &2.310(6)     &0.45(1)    &\multicolumn{1}{c}{-} &\multicolumn{1}{c}{-}      &-134(28)   &0.7/3\\ 

16    &2.288(5)     &0.52(1)    &-2.8(4)    &4.9(9)     &\multicolumn{1}{c}{-}      &3.8/6\\ 
24    &2.28(1)      &0.53(3)    &-3(1)      &7(3)       &\multicolumn{1}{c}{-}      &3.4/5\\ 

16    &2.297(4)     &0.490(10)  &-1.7(2)    &\multicolumn{1}{c}{-}       &13(2)      &4.5/6\\ 
24    &2.291(8)     &0.51(2)    &-2.0(4)    &\multicolumn{1}{c}{-}       &21(9)      &3.7/5\\ 
\hline 
\end{tabular} 
\vspace{3ex}
\end{table} 

\begin{figure}[t]
	\centering
	\includegraphics[width=0.5\columnwidth]{figures/C1_C2_3D.pdf}
	\caption{The scaling behaviors of $C_1(w_c, L)$ and $C_2(w_1, L)$ on the cubic lattice, plotted on a log-log scale. The dashed lines
	represent $C = a_0 L^{d_{\text{f}}}$ with $d_\text{f} = 2.5838$ for the black one and $d_{\text{f}} =2.30$ for the blue one.}
	\label{fig:C1_C2_3D}
\end{figure}


\begin{figure}[b]
	\vspace{3ex}
	\centering
	\includegraphics[width=0.85\columnwidth]{figures/Prob_w2_3D.pdf}
	\caption{The semi-log plot of the probability density functions $f_{X_1}(x)$ and $f_{X_2}(x)$ at $w_1$ for various system sizes, 
	where $X_1 \equiv (\scrC_1 - C_1)/L^{2.3}$ and $X_2 \equiv \scrC_2/L^{2.3}$.}
	\label{fig:prob_w2_3D}
	\vspace{3ex}
\end{figure}

We then study the probability distribution of $\scrC_1$ and $\scrC_2$. Figure~\ref{fig:prob_w2_3D} shows the probability density 
functions of variables $X_1 \equiv \frac{\scrC_1 - C_1}{L^{2.3}}$ and $X_2 \equiv \frac{\scrC_2}{L^{2.3}}$ for various system sizes. 
Similar to the case of CG, the data exhibit good collapse only for a partial region, which corresponds to the sector in which
the fluctuations of $\scrC_1$ are of the scale $O(L^{2.3})$ and $C_2 \sim L^{2.3}$. The failure of collapsing the whole curve
of $f_{X_1}(x_1)$ and $f_{X_2}(x_2)$ implies that there is another sector with distinct critical behaviors.
The conjecture is confirmed by Fig.~\ref{fig:prob_w2_3D_tail}, where we define $X_1^\prime \equiv \frac{\scrC_1 - C_1}{L^3}$ and $X_2^\prime \equiv \frac{\scrC_2 - C_2}{L^3}$.
A good data collapse is observed for $L^{0.75}f_{X_1^\prime}(x)$ and $L^{0.75}f_{X_1^\prime}(x)$ in the region where the former sector fails to collapse.
Therefore, the sector vanishes asymptotically as $L^{-0.75}$. In contrast to the former sector, the largest cluster in this sector has $O(L^3)$ fluctuations.
For convenience, we denote the two sectors as sector~\Romannum{1} and sector~\Romannum{2}.


\begin{figure}[t]
	\vspace{3ex}
	\centering
	\includegraphics[width=0.85\columnwidth]{figures/Prob_tail_w2_3D.pdf}
	\caption{The semi-log plot of the probability density functions $f_{X_1^\prime}(x)$ and $f_{X_2^\prime}(x)$ for various system sizes, 
	where $X_1^\prime \equiv (\scrC_1 - C_1)/L^{3}$ and $X_2^\prime \equiv (\scrC_2 - C_2)/L^{3}$.}
	\label{fig:prob_w2_3D_tail}
	\vspace{3ex}
\end{figure}

We further study the cluster-size distribution $n^\prime(s, L)$ of the non-giant clusters in the two sectors. To avoid finite-size corrections
as much as possible, we make measurements for the sector~\Romannum{1} only when $\scrC_1 - C_1 > 0$ and for the sector~\Romannum{2} only when $\scrC_1 - C_1 < -0.1V$.
The results are shown in Fig.~\ref{fig:nsp_w2_3D}. For sector~\Romannum{1}, a power-law behavior $n^\prime(s, L) \sim s^{-\tau}$ is observed
for $s \ll L^{\dtf}$ in Fig.~\ref{fig:nsp_w2_3D}(a), and the Fisher exponent $\tau = 2.30$ is consistent with the one obtained via the formula $1+d/\dtf$ using our estimate $\dtf=2.30(3)$.
Moreover, the inset in Fig.~\ref{fig:nsp_w2_3D}(a) suggests that the distribution can be written as $n^\prime(s, L) = s^{-\tau}\tilde{n}^\prime(s/L^{\dtf})$,
which is the same scaling form for the total distribution $n(s, L)$ at criticality, except that the values of exponents now are different.
For sector~\Romannum{2}, the same power-law decay is also observed, as indicated by the black dashed line in Fig.~\ref{fig:nsp_w2_3D}(b). Besides, we find a scaling
$n^\prime(s, L) \sim s^{-2}$ when $s$ is of the order of $V$. In the inset of Fig.~\ref{fig:ns_wc}(b), we plot $s^2 n^\prime(s, L)$ versus $s/L^3$. The data
seems not to collapse very well for $s \sim V$, which is probably due to the finite-size corrections and the insufficient sample numbers for this region.
\begin{figure}[H]
	\centering
	\includegraphics[width=0.85\columnwidth]{figures/nsp_w2_3D.pdf}
	\caption{The log-log plot of the cluster-size distribution $n^\prime(s, L)$ on the cubic lattice sampled in (a) sector~\Romannum{1} and (b) sector~\Romannum{2}. 
	The insets show the scaled distribution $s^\tau n^\prime(s, L)$ versus $s/L^{d_\text{f}}$, 
	where the values of $\tau$ and $d_\text{f}$ are indicated in the legends.}
	\label{fig:nsp_w2_3D}
\end{figure}




